%% LaTeX2e class for seminar theses
%% sections/abstract_en.tex
%% 
%% Karlsruhe Institute of Technology
%% Institute for Program Structures and Data Organization
%% Chair for Software Design and Quality (SDQ)
%%
%% Dr.-Ing. Erik Burger
%% burger@kit.edu

Asynchronous communication is becoming increasingly popular in microservice software systems due to providing scalable and resilient distributed applications.
In a continuous software engineering environment where software is subject to erosion, we apply reverse engineering approaches to extract the architecture of the system.
In this systematic literature review, we compare five different approaches to extract the architecture of such asynchronous microservice software systems.
We will provide an overview of the current state of research in the field and classify the approaches using the five aspects input, approach, output, end user and evaluation.
We conclude that three out of the five approaches are capable of extracting message-based asynchronous communication as well as HTTP RESTful asynchronous communication, while the other two approaches only support HTTP RESTful asynchronous communication.
We determine the MiSAR approach as the most flexible approach with the drawback of being a manual approach.
The approach by Mayer and Weinreich aggregates the data and presents it nicely in a dashboard that is easy to understand.
The approach by Brosig on the other hand is able to extract full Palladio Component Models, while being fully automated.