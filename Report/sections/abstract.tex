%% LaTeX2e class for seminar theses
%% sections/abstract_en.tex
%% 
%% Karlsruhe Institute of Technology
%% Institute for Program Structures and Data Organization
%% Chair for Software Design and Quality (SDQ)
%%
%% Dr.-Ing. Erik Burger
%% burger@kit.edu

Asynchronous communication is becoming increasingly popular in microservice software systems.
It provides scalable and resilient distributed applications.
In a continuous software development process, software architectures are subjected to erosion.
To understand and refactor the software architecture at any phase of the development cycle, a software architect requires the architecture.
Reverse engineering approaches support architecture extraction for such systems.
In this systematic literature review, we compare five different approaches to extract the architecture of such asynchronous microservice software systems.
We provide an overview of the current state of research in the field and classify the approaches using five different aspects, such as: input, approach, output, end user and evaluation.
We conclude that three out of the five approaches are capable of extracting message-based asynchronous communication as well as HTTP RESTful asynchronous communication, while the other two approaches only support HTTP RESTful asynchronous communication.