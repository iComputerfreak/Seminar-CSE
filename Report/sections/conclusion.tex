%% LaTeX2e class for seminar theses
%% sections/conclusion.tex
%% 
%% Karlsruhe Institute of Technology
%% Institute for Program Structures and Data Organization
%% Chair for Software Design and Quality (SDQ)
%%
%% Dr.-Ing. Erik Burger
%% burger@kit.edu

\section{Conclusion and Future Work}
\label{cha:Conclusion}
In this paper, we compared five approaches that each are capable of extracting the architecture of an asynchronous microservice software system.
We laid out some foundations and our selection of approaches and then proceeded with the results.
We compared the five approaches in regards to their input, the technique they used, their output, the end user they were designed for and the type of evaluation the authors used.

We showed that in terms of flexibility, MiSAR \cite{Alshuqayran2018MiSAR} takes the lead, being the most flexible with the drawback of being a manual approach.
In terms of output, the approach by Brosig \cite{Brosig2011} extracts a full Palladio Component Model, including the usage data, allowing software architects to perform software quality predictions.
Mayer and Weinreich's approach \cite{Mayer2018} on the other hand displays the extracted and aggregated data nicely in a dashboard so a normal software practitioner can use it.

Two other approaches that are not discussed in this paper are MicroART \cite{Granchelli2017MicroART} and the approach by Langhammer et al. \cite{Langhammer2016}.
These approaches are not included in the comparison, because they do not match the inclusion criteria in \autoref{table:InclusionExclusion}.
Future work could focus on making these two approaches compatible with asynchronous systems.

Additionally, future work could focus on improving the approaches presented here.
ARCHI4MOM could be extended to support mixed-technology systems (i.e. systems that include both synchronous as well as asynchronous communication).

In general, there exists very little work outside of the presented approaches about the extraction of the architecture of asynchronous microservice software systems, especially for systems using message-based communication, motivating it as a research topic for the future. 