%% LaTeX2e class for seminar theses
%% sections/content.tex
%% 
%% Karlsruhe Institute of Technology
%% Institute for Program Structures and Data Organization
%% Chair for Software Design and Quality (SDQ)
%%
%% Dr.-Ing. Erik Burger
%% burger@kit.edu

\section{Foundation}
\label{ch:Foundation}

%% -------------------
%% | Example content |
%% -------------------
General foundation knowledge.

\subsection{Microservice Architecture}
Explanation of microservice architectures.
\cite{Dragoni2017}

\subsection{Architecture Extraction}
\label{sec:FirstContentSection:FirstSubSubSection}
Explanation of Architecture Extraction.

\subsection{PCM}
\label{sec:FirstContentSection:FirstSubSection}
Explanation of the Palladio Component Model.
\cite{Becker2008}

\section{Study Design}
\label{sec:SecondContentSection:FirstSubsection}
Explain how the literature research is done.

\subsection{Study Aim}
\label{sec:SecondContentSection:SecondSubsection}
Explain the goal of the paper.

\subsection{Research Questions}
The research questions I want to to answer in this paper.

\subsection{Selecting the Papers}
Explanation of how I searched for the papers. The keywords I used and the selection criteria (inclusion/exclusion).

Async Papers:
\begin{enumerate}
	\item \cite{Singh2022ARCHI4MOM}, \cite{Singh2021} (MOM)
	\item \cite{Alshuqayran2018MiSAR} (MOM)
	\item \cite{Brosig2011} (MOM)
	\item \cite{Mayer2018} (HTTP)
	\item \cite{Kleehaus2018} (HTTP)
	\item \cite{Ntentos2021} (HTTP)
\end{enumerate}

\section{Results}
Answers to the research questions.
Analysis of the papers I found.

\section{Discussion}
Not sure if I need this section.

\section{Threats to Validity}
Explain the threats to internal and external validity of the study.

\section{Related Work}
Other papers that are similar to my paper.
Also say that e.g. MicroART could perhaps support async communication in the future.
\cite{Granchelli2017MicroART}
\cite{Ducasse2009}



%Add additional content sections if required by adding new .tex files in the
%\code{sections/} directory and adding an appropriate 
%\code{\textbackslash input} statement in \code{thesis.tex}. 
%% ---------------------
%% | / Example content |
%% ---------------------