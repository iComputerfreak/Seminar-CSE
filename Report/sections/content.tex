%% LaTeX2e class for seminar theses
%% sections/content.tex
%% 
%% Karlsruhe Institute of Technology
%% Institute for Program Structures and Data Organization
%% Chair for Software Design and Quality (SDQ)
%%
%% Dr.-Ing. Erik Burger
%% burger@kit.edu

\section{Foundation}
\label{cha:Foundation}
This chapter will provide some foundation knowledge for the rest of the paper.

\subsection{Microservice Architecture}
\label{sec:Foundation:MicroserviceArchitecture}
This section explains the general concepts of a microservice architecture.
\cite{Dragoni2017}

\subsection{Asynchronous Communication}
\label{sec:Foundation:AsyncCommunication}
This section will explain the basics of asynchronous communication.

\subsection{Message-based Communication}
\label{sec:Foundation:MessageBasedCommunication}
This section will explain what message-based communication is in particular.

\subsection{Software Architecture Extraction (SAR)}
\label{sec:Foundation:SAR}
This section will explain Software Architecture Extraction (SAR) in general.

\subsection{Palladio Component Model (PCM)}
\label{sec:Foundation:PCM}
This section provides foundation knowledge of the Palladio Component Model (PCM).
\cite{Becker2008}

\section{Study Design}
\label{cha:StudyDesign}
This chapter will explain the design of the systematic literature research and how it was executed.

\subsection{Study Aim}
\label{sec:StudyDesign:StudyAim}
This section explains the goal of the paper, namely to compare different tools for SAR for asynchronous communication-based systems.

\subsection{Research Questions}
\label{sec:StudyDesign:ResearchQuestions}
This section presents the research questions we will answer in the following sections.

The research questions are: \\
\begin{itemize}
	\item \textbf{RQ1}. What are the tools available for extraction of asynchronous architecture?
	\item \textbf{RQ2}. To what extend do the tools support software architecture extraction?
\end{itemize}

\subsection{Selecting the Papers}
\label{sec:StudyDesign:SelectingPapers}
This section explains how the papers were selected, including the keywords used to search for papers and the selection criteria (inclusion/exclusion).

Asynchronous Papers found:
\begin{enumerate}
	\item ARCHI4MOM \cite{Singh2022ARCHI4MOM}, \cite{Singh2021}
	\item MiSAR \cite{Alshuqayran2018MiSAR}
	\item --- \cite{Brosig2011}
	\item MYCROLYZE \cite{Kleehaus2018} (only supports asynchronous HTTP communication)
	\item --- \cite{Mayer2018} (only supports asynchronous HTTP communication)
	\item --- \cite{Ntentos2021} (only supports asynchronous HTTP communication)
\end{enumerate}

\section{Results}
\label{cha:Results}
This chapter will answer the research questions formulated in \ref{sec:StudyDesign:ResearchQuestions} and analyze the papers selected in \ref{sec:StudyDesign:SelectingPapers}.
This chapter will also feature a table comparing the papers in different aspects:
\begin{enumerate}
	\item \textbf{Input} (e.g. source code, artifacts, logs, ...)
	\item \textbf{Output} (e.g. PCM, UML, ...)
	\item \textbf{Approach} (how the tools extract the architecture)
	\item \textbf{End user} (who the result is intended for)
	\item \textbf{Evaluation metric} (how were the results evaluated, e.g. precision/recall or comparison)
\end{enumerate}

\section{Discussion}
\label{cha:Discussion}
This chapter will discuss the results of the previous chapter.

\section{Related Work}
\label{cha:RelatedWork}
This chapter presents other papers which are similar to my work.
For example \cite{Ducasse2009} ...
We will also talk about the fact that \cite{Granchelli2017MicroART} and \cite{Langhammer2016} could be extended to support asynchronous communication in the future.




%Add additional content sections if required by adding new .tex files in the
%\code{sections/} directory and adding an appropriate 
%\code{\textbackslash input} statement in \code{thesis.tex}. 
%% ---------------------
%% | / Example content |
%% ---------------------