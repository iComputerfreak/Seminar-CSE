%% LaTeX2e class for seminar theses
%% sections/content.tex
%% 
%% Karlsruhe Institute of Technology
%% Institute for Program Structures and Data Organization
%% Chair for Software Design and Quality (SDQ)
%%
%% Dr.-Ing. Erik Burger
%% burger@kit.edu

Keywords

\section{Introduction}
\label{cha:Introduction}
This chapter will provide a motivation of why a systematic literature review about tools for architecture extraction of asynchronous systems is necessary.

In chapter 2, we will provide some foundation knowledge.
In chapter 3, we will talk about the design and goal of this literature review and the selection of the papers.
In chapter 4, we will present the results and compare them using five different aspects.
We will discuss these results in chapter 5 and talk about related work in chapter 6, before drawing conclusions in chapter 7.

%% -------------------
%% | Example content |
%% -------------------

%This is the SDQ seminar template.
%For more information on the formatting of theses at SDQ, please refer to
%\url{https://sdqweb.ipd.kit.edu/wiki/Ausarbeitungshinweise} or to your advisor.
%
%Test: ``This is a quote''.
%
%\subsection{Example: Citation}
%\label{sec:Introduction:Citation}
%A citation: \cite{becker2008a} For referencing, see \autoref{sec:Introduction:Figures}
%
%\subsection{Example: Figures}
%\label{sec:Introduction:Figures}
%\begin{figure}
%\centering
%\includegraphics[width=4cm]{images/sdqlogo}
%\caption{SDQ logo}
%\label{fig:sdqlogo}
%\end{figure}
%
%A reference: The SDQ logo is displayed in \autoref{fig:sdqlogo}. 
%(Use \code{\textbackslash autoref\{\}} for easy referencing.) 
%
%\subsection{Example: Tables}
%\label{sec:Introduction:Tables}
%\begin{table}
%\centering
%\begin{tabular}{r l}
%\toprule
%abc & def\\
%ghi & jkl\\
%\midrule
%123 & 456\\
%789 & 0AB\\
%\bottomrule
%\end{tabular}
%\caption{A table}
%\label{tab:atable}
%\end{table}
%
%\subsection{Example: Todo-Note}
%Meaningless text.
%
%\subsection{Example: Formula}
%One of the nice things about the Linux Libertine font is that it comes with
%a math mode package.
%\begin{displaymath}
%f(x)=\Omega(g(x))\ (x\rightarrow\infty)\;\Leftrightarrow\;
%\limsup_{x \to \infty} \left|\frac{f(x)}{g(x)}\right|> 0
%\end{displaymath}

%% --------------------
%% | /Example content |
%% --------------------