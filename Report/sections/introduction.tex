%% LaTeX2e class for seminar theses
%% sections/content.tex
%% 
%% Karlsruhe Institute of Technology
%% Institute for Program Structures and Data Organization
%% Chair for Software Design and Quality (SDQ)
%%
%% Dr.-Ing. Erik Burger
%% burger@kit.edu

\section{Introduction}
\label{cha:Introduction}
Asynchronous microservices communicate with each other through messages.
The architecture of an asynchronous microservice system plays a crucial role in its design, development, and maintenance.
It determines how the services are organized, how they interact with each other and with the outside world, and how they can evolve and adapt over time.

In the context of continuous software engineering, where software is developed and deployed continuously, the architecture of an asynchronous microservice system is subjected to erosion, i.e., the gradual divergence between the intended architecture and the implemented architecture over time \cite{DESILVA2012132}.
This can happen due to various factors, such as changes in the requirements, the environment, the technology, or the team.
One solution to this problem is to apply a Reverse Engineering approach to get the up-to-date architecture.
The extracted architecture can be used by the software architect to refactor the software system.

One way to understand the architecture of an asynchronous microservice system is by extracting it from the source code and other artifacts that describe the system.
Using static code analysis, artifacts are statically analyzed to detect the microservice components and the communication between them.
Another way to extract the architecture is dynamic analysis, where a software architect executes the system and observes the runtime behaviour, logs and other data to detect the architecture of the software-system.

In this paper, we compare five reverse engineering approaches that are capable of extracting the architecture of asynchronous microservice systems.
In \autoref{cha:Foundation}, we provide the foundation on which our work builds on.
\autoref{cha:StudyDesign} describes the goal of this literature review and the selection criteria of the papers.
In \autoref{cha:Results}, we classify the results by five different aspects.
We discuss these results in \autoref{cha:Discussion}, followed by the related work in \autoref{cha:RelatedWork}.
\autoref{cha:Conclusion} concludes the results and discuss future work.
