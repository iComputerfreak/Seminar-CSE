%% LaTeX2e class for seminar theses
%% sections/content.tex
%% 
%% Karlsruhe Institute of Technology
%% Institute for Program Structures and Data Organization
%% Chair for Software Design and Quality (SDQ)
%%
%% Dr.-Ing. Erik Burger
%% burger@kit.edu

\section{Introduction}
\label{cha:Introduction}
Asynchronous microservices communicate with each other, e.g. through messages or HTTP-communication.
The architecture of an asynchronous microservice system plays a crucial role in its design, development, and maintenance.
It determines how the services are organized, how they interact with each other and with the outside world, and how they can evolve and adapt over time.

In the context of continuous software engineering, where software is developed and deployed continuously, the architecture of an asynchronous microservice system is subject to erosion, i.e., the gradual divergence between the intended architecture and the implemented architecture over time \cite{DESILVA2012132}.
This can happen due to a variety of factors, such as changes in the requirements, the environment, the technology, or the team.
To get the up-to-date architecture, we apply reverse engineering approaches to extract the architecture of the system.

One way to understand the architecture of an asynchronous microservice system is by extracting it from the codebase and other artifacts that describe the system.
This can be achieved by static and dynamic analysis.
Using static code analysis, artifacts are statically analyzed to detect the microservice components and the communication between them.
Another way to extract the architecture is dynamic analysis, where we execute the system and observe the runtime behaviour, logs and other data to detect the architecture of the software-system.

In this paper, we compare five reverse engineering approaches that are capable of extracting the architecture of asynchronous microservice systems.
In \autoref{cha:Foundation}, we provide some foundation knowledge.
In \autoref{cha:StudyDesign}, we talk about the goal of this literature review and the selection of the papers.
In \autoref{cha:Results}, we present the results and compare them using five different aspects.
We discuss these results in \autoref{cha:Discussion}, followed by the related work in \autoref{cha:RelatedWork}.
\autoref{cha:Conclusion} concludes our results.
