\autoref{table:results} shows the results of the comparison in tabular format.
In the following sections, we will talk about each paper individually.

Each approach will be presented using five aspects:
\begin{enumerate}
	\item \textbf{Input} (e.g. source code or logs)
	\item \textbf{Approach} (how the extraction process works)
	\item \textbf{Output} (e.g. PCM or UML)
	\item \textbf{End user} (who the result is intended for)
	\item \textbf{Evaluation} (how were the results evaluated)
\end{enumerate}

\subsection{ARCHI4MOM}
\label{sec:Results:ARCHI4MOM}
\textbf{Input.}
The ARCHI4MOM approach extends the Performance Model Extraction (PMX) approach \cite{Walter2017PMX,Singh2022ARCHI4MOM} and therefore takes the same inputs.
The first step in the ARCHI4MOM approach is to instrument the source code with the Jaeger tracing tool \footnote{https://www.jaegertracing.io/} to collect tracing data.
Using the OpenTracing API, ARCHI4MOM then instruments all microservices to generate trace data, which can be used later to reconstruct the asynchronous communication between the microservices. \cite{Singh2022ARCHI4MOM} % TODO: Also supports MOM
\\ \\
\textbf{Approach.}
ARCHI4MOM extends the Performance Model Extraction (PMX) approach \cite{Walter2017PMX,Singh2022ARCHI4MOM} to support asynchronous communication.
This is achieved by adding a dependency to the OpenTracing API to each microservice to introduce a new set of information that was not present earlier \cite{Singh2022ARCHI4MOM}.
This tracing data is then collected in the form of JavaScript Object Notation (JSON) files, which become the input of the next phase \cite{Singh2022ARCHI4MOM}.

The JSON files will then be used to analyze the structure of the traces.
For message-based asynchronous communication, this presents a challenge since the information is distributed over different \textit{spans} whereas using synchronous communication, it would be in a single span \cite{Singh2022ARCHI4MOM}.
To match this inter-component communication using middleware, the called method needs to be matched using the OpenTracing API tags \cite{Singh2022ARCHI4MOM}.
In the next step, the ARCHI4MOM approach looks for send operations that do not have information about the topic \todo{explain} they send to and fills in this information by retrieving it from other send operations in the tracing data \cite{Singh2022ARCHI4MOM}.
Then the approach iterates over all sending spans that have a FOLLOWS-FROM or \textit{message-bus} relation \todo{explain} tag and propagate their topics to the receiving spans \cite{Singh2022ARCHI4MOM}.

To reconstruct message-based communication using PMX, the authors extended PMX to support the required Palladio Component Model elements \footnote{https://github.com/PalladioSimulator/Palladio-Addons-Indirections/tree/maste r/bundles/org.palladiosimulator.indirections/model}.
The authors then implement additional PMX logic to be able to reconstruct asynchronous architectures using these model elements.
\cite{Singh2022ARCHI4MOM}.
\\ \\
\textbf{Output.}
The output of the ARCHI4MOM approach is a Palladio Component Model (PCM) \cite{Singh2022ARCHI4MOM}.
This PCM contains the extracted components, as well as the interfaces for communication between each other \cite{Singh2022ARCHI4MOM}.
The communication channels are represented by \textit{DataChannel}s (representing the middleware) and \textit{DataInterface}s (representing the type of data the interface can send/receive), which are created in the PCM repository as part of the extraction \cite{Singh2022ARCHI4MOM}.
\\ \\
\textbf{End User.}
The output of the ARCHI4MOM approach is a PCM.
This model can then be used by software architects together with usage scenarios to create simulations, predicting the non-functional properties of the software.
\\ \\
\textbf{Evaluation.}
To evaluate the approach, the authors created a manual PCM of the Flowing Retail sample application \footnote{https://github.com/berndruecker/flowing-retail/tree/master/kafka/java}.
This manual model was then verified by three developers to be correct and compared to the automatically extracted model using a Goal Question Metric (GQM) plan \cite{VanSolingen2002GQM,Singh2022ARCHI4MOM}.
Using this plan, both sets of model elements (manual and automatic) are then compared using Precision, Recall and F1 score \cite{Singh2022ARCHI4MOM}.
The automatic approach achieved a precision score of 100\%, a recall score of 95.65\% and an F1 score of 97.8\% \cite{Singh2022ARCHI4MOM}. \todo{should I include the results or not?}

\subsection{MiSAR}
\label{sec:Results:MiSAR}
\textbf{Input.}
MiSAR extracts and gathers different data from static artifacts.
This data includes docker files that assemble the containers for the microservices, docker compose files that orchestrate multi-docker-container systems, java source code, maven pom.xml files, YAML configuration files, documentation and tool support \cite{Alshuqayran2018MiSAR}.
The java source code is reverse engineered using a tool called Enterprise architect \footnote{http://www.sparxsystems.com.au/products/ea/}, providing UML class diagrams from the source code \cite{Alshuqayran2018MiSAR}.
Additionally, Zipkin \footnote{https://zipkin.io} is used to trace communication between microservices to build a call graph \cite{Alshuqayran2018MiSAR}.
Information about latencies was retrieved using TCPDump \footnote{https://www.tcpdump.org} and information about the ports, IP addresses of container, and connectivity between containers were extracted using the Sysdig tool \footnote{https://github.com/draios/sysdig} \cite{Alshuqayran2018MiSAR}.
This information is all stored in a repository for further use.
\\ \\
\textbf{Approach.}
% TODO: Explain MDE, PIM, PSM; approach transforms from PSM to PIM using mapping rules
MiSAR is a manual approach that is executed in two phases \cite{Alshuqayran2018MiSAR}.
The first phase (Recovery Design, RD) defines architectural concepts, which are extracted in the second phase (Recovery Execution, RE) \cite{Alshuqayran2018MiSAR}.
\todo{extend}
\\ \\
\textbf{Output.}
\\ \\
\textbf{End User.}
\\ \\
\textbf{Evaluation.}

\subsection{--- \cite{Brosig2011}}
\label{sec:Results:Brosig}
\textbf{Input.}
The approach by Brosig et al. uses onlyl monitoring data collected at runtime to extract the effective architecture \cite{Israr2007interaction} of the system.
\\ \\
\textbf{Approach.}
% only considers parts effectively used at runtime
The approach presented by Brosig et al. only extracts the effective architecture \cite{Israr2007interaction} of the system, meaning that only parts that are effectively used at runtime are considered \cite{Brosig2011}.
The first step in the extraction of the effective architecture.
Before the extraction process can begin, component boundaries, which separate components as single entities from the point of view of the system's architect, need to be defined \cite{Brosig2011}.
This can be either done manually by a software architect or automatically using static code analysis \cite{Brosig2011}.
After the system boundaries have been determined, the running system is monitored using \textit{call path tracing} \cite{Brosig2011}.
The resulting \textit{event records} (representing entries or exists of components) are grouped together into \textit{call path event record sets} which contain event records that were triggered by the same system request \cite{Brosig2011}.
This data can then be used to obtain a call path \cite{Brosig2011} as well as a list of external services, a component's provided interface calls.
To extract an accurate representation of the system's components and connections between them, a representative usage profile has to be chosen, as the extraction only captures the actual communication that happens during the extraction approach \cite{Brosig2011}.
After the components and their connection have been extracted, the component-internal performance-relevant control flow has to be modeled.
This includes the internal behavior as well as the external service calls, the component makes \cite{Brosig2011}.

For the second step, the approach aims to extract model parameters for performance prediction \cite{Brosig2011}.
This is achieved by extracting branch probabilities and loop iteration numbers from the call paths \cite{Brosig2011}.
For branching probabilities, mean values are used, whereas loop iteration numbers are represented by a Probability Mass Function (PMF) to allow for accurate representation of cases where a loop is for example either executed twice or ten times \cite{Brosig2011}.
Next, the approach tries to quantify the resource demands (e.g. CPU, HDD) of the components' internal computations \cite{Brosig2011}.
This demand is represented by the total processing time minus the time spent waiting for the resource to become available \cite{Brosig2011}.
These parameters are then averaged over the observed call paths \cite{Brosig2011}.
The approach is also able to handle e.g. branches that depend on input parameter values \cite{Brosig2011}.
In this case, the dependency has to be known a-priori for the approach to quantify these dependencies \cite{Brosig2011}.

In the third step, the performance model is calibrated by comparing its predictions with measurements on the real system \cite{Brosig2011}.
The correction to be done when measuring a deviation between the prediction and the measurement is done by increasing a factor of overhead and accounting for delays produced by the middleware stack, the system runs on \cite{Brosig2011}.
% Continue reading Proof of Concept for more infos
\\ \\
\textbf{Output.}
% Output is PCM
The output of the approach by Brosig et al. is a performance model \cite{Brosig2011}.
In their proof-of-concept implementation, they generate a Palladio Component Model (PCM) \cite{Brosig2011}.
\\ \\
\textbf{End User.}
The end user for this approach is e.g. a software architect, which uses the performance model to predict software quality attributes.
\\ \\
\textbf{Evaluation.}
The evaluation of the approach was done by implementing it and applying it to a case study of a Java Enterprise Edition application \cite{Brosig2011}.
The application used was the SPECjEnterprise2010 benchmark \footnote{https://www.spec.org/jEnterprise2010/}.


\subsection{MICROLYZE \cite{Kleehaus2018}}
\label{sec:Results:Microlyze}
\textbf{Input.}
\\ \\
\textbf{Approach.}
\\ \\
\textbf{Output.}
\\ \\
\textbf{End User.}
\\ \\
\textbf{Evaluation.}

\subsection{--- \cite{Mayer2018}}
\label{sec:Results:Mayer}
\textbf{Input.}
\\ \\
\textbf{Approach.}
\\ \\
\textbf{Output.}
\\ \\
\textbf{End User.}
\\ \\
\textbf{Evaluation.}

% TODO: Answer RQ!



% TODO: Table has to be in horizontal orientation
% TODO: Header should be bold
% TODO: Line breaks in cells
% TODO: table too wide

\begin{sidewaystable}
\centering
% check that total width matches
\resizebox{\textwidth}{!}{
%                 Name       Input    Approach Output   User     Eval     Year     Type
\begin{tabular}{| p{2.5cm} | p{2cm} | p{4cm} | p{2cm} | p{1.5cm} | p{4cm} | p{1cm} | p{1.2cm} |}
\toprule
\textbf{Name} & \textbf{Input} & \textbf{Approach} & \textbf{Output} & \textbf{End User} & \textbf{Evaluation} & \textbf{Year} & \textbf{Type} \\
\midrule
ARCHI4MOM (\cite{Singh2022ARCHI4MOM})
& source code
& Extend PMX to support asynchronous architectures
& PCM
& software architect
& Comparison with manual architecture
& 2022
& tool \\
\midrule
MiSAR (\cite{Alshuqayran2018MiSAR})
& source code, descriptive files, runtime traces
& manual extraction approach \todo{extend}
& 
& 
& 
& 2018
& manual approach \\
\midrule
--- (\cite{Brosig2011})
& run-time monitoring data
& combine an existing call path tracing and resource demand estimation to an end-to-end model extraction
& PCM
& 
& SPECjEnterprise2010 benchmark application; comparison of prediction with measurements
& 2011
& tool \\
\midrule
MICROLYZE (\cite{Kleehaus2018}) 
& monitoring data
& continuously monitor for system changes using a service discovery service; trace HTTP requests using zipkin
& database with services and their relations; web application to visualize as adjacency matrix
& system administrators and enterprise or software architects
& approach was applied to TUM LLCM platform, Travelcompanion service; traffic was generated and result was manually checked
& 2018
& tool \\
\midrule
--- (\cite{Mayer2018}) 
& static service information, infrastructure information and runtime logs
& condense static and dynamic information into single dimension to analyze the evolution over time; visualize information in dashboard
& aggregated data, visualized in dashboard
& 
& use tool in testing environment; check viability of results
& 2018
& tool \\
\bottomrule
\end{tabular}
}
\caption{Results} % TODO: Explanation text
\label{table:results}
\end{sidewaystable}
