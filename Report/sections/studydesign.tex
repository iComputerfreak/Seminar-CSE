In this section, we discuss the design of the systematic literature review and how we execute it.

\subsection{Study Aim}
\label{sec:StudyDesign:StudyAim}
A systematic literature review is a means of identifying, evaluating and interpreting all available research relevant to a particular research question, or topic area, or phenomenon of interest \cite{Keele2007guidelines}.
The aim of this paper is to find and compare the tools available for the extraction of the architecture of asynchronous microservice software-systems.
Our search criteria is followed by two guiding questions.

\textbf{Q1.} What are the tools available for the extraction of asynchronous architectures of microservice systems?

\textbf{Q2.} To what extend do these tools support asynchronous software architecture extraction?

\subsection{Selecting the Papers}
\label{sec:StudyDesign:SelectingPapers}
We searched several research databases (e.g., Google Scholar) to collect the papers.
In order to perform the search, we used several combinations of strings as mentioned below.
However, this is not the exhaustive set of the strings.
The following queries are used to search for papers:
\begin{itemize}
	\item \code{architecture (extraction OR reconstruction) (dynamic OR logs OR \\
 asynchronous) microservice}
	\item \code{("architecture extraction" OR "architecture reconstruction") (dynamic OR logs OR asynchronous) microservice}
	\item \code{reverse engineering (dynamic OR logs OR asynchronous) (microservice OR \\
 mixed-technology)}
\end{itemize}

Additionally, the references of the found results were used to look for further papers.
We select the papers based on the criteria mentioned in \autoref{table:InclusionExclusion}.


\begin{table}
\centering
\begin{tabular}{r l}
\toprule
Inclusion
& Papers that present an approach for extracting asynchronous \\
& microservice architectures \\
\midrule
Exclusion
& Papers that only present a foundation or compare other approaches \\
& for such systems \\
\bottomrule
\end{tabular}
\caption{Inclusion and exclusion criteria for selecting the papers}
\label{table:InclusionExclusion}
\end{table}


%% Inclusion
%% - micro service architectures
%% - must support extraction of asynchronous communication
%% - must present an approach to extract the architecture
%% (also keep papers that only provide a foundation/framework?) - no

In total, we will look at five papers, that each present an approach for the extraction of asynchronous architectures of microservice systems.
Since only three of the papers have named their approach, we choose to refer to the other two approaches by their authors' names.

The papers are
\begin{enumerate}
	\item ARCHI4MOM \cite{Singh2022ARCHI4MOM}, \cite{Singh2021}
	\item MiSAR \cite{Alshuqayran2018MiSAR}
	\item Approach by Brosig et al. \cite{Brosig2011}
	\item MICROLYZE \cite{Kleehaus2018} % (only supports asynchronous HTTP communication)
	\item Approach by Mayer and Weinreich \cite{Mayer2018} % (only supports asynchronous HTTP communication)
\end{enumerate}