This chapter will explain the design of the systematic literature review and how it was executed.

\subsection{Study Aim}
\label{sec:StudyDesign:StudyAim}
\todo{Allgemein: was ist SLR?}
The aim of this paper is to find and compare the tools available for the extraction of the architecture of asynchronous microservice systems.
For this purpose, we define two guiding questions.

\textbf{Q1.} What are the tools available for the extraction of asynchronous architectures of microservice systems?

\textbf{Q2.} To what extend do the tools support software architecture extraction?

\subsection{Selecting the Papers}
\label{sec:StudyDesign:SelectingPapers}
To search for research papers, I (\todo{``we'' or ``I'' here?} performed several queries using Google Scholar.
The following queries were used to search for papers:
\begin{itemize}
	\item \code{architecture (extraction OR reconstruction) (dynamic OR logs OR asynchronous) microservice}
	\item \code{("architecture extraction" OR "architecture reconstruction") (dynamic OR logs OR asynchronous) microservice}
	\item \code{reverse engineering (dynamic OR logs OR asynchronous) (microservice OR mixed-technology)}
\end{itemize}

Additionally, the references of the found results were used to look for further papers.
For a paper to be selected, it had to match our selection criteria depicted in \autoref{table:InclusionExclusion}


% TODO: Vertical separators
\begin{table}
\centering
\begin{tabular}{r l}
\toprule
Inclusion
& Papers that present an approach for extracting microservice architectures \\
& Papers that are able to extract asynchronous architectures \\
\midrule
Exclusion
& Papers that only present a foundation or compare other approaches \\
\bottomrule
\end{tabular}
\caption{Inclusion and exclusion criteria for selecting the papers}
\label{table:InclusionExclusion}
\end{table}


%% Inclusion
%% - micro service architectures
%% - must support extraction of asynchronous communication
%% - must present an approach to extract the algorithm
%% (also keep papers that only provide a foundation/framework?)

In total, we will look at five papers, which each present an approach for the extraction of asynchronous architecture of microservice systems.


The papers are
\begin{enumerate}
	\item ARCHI4MOM \cite{Singh2022ARCHI4MOM}, \cite{Singh2021}
	\item MiSAR \cite{Alshuqayran2018MiSAR}
	\item Approach by Brosig et al. \cite{Brosig2011}
	\item MICROLYZE \cite{Kleehaus2018} % (only supports asynchronous HTTP communication)
	\item Approach by Mayer and Weinreich \cite{Mayer2018} % (only supports asynchronous HTTP communication)
\end{enumerate}